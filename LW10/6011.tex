\documentclass[11pt]{extbook}
\usepackage[utf8]{inputenc}
\usepackage[russian]{babel}
\usepackage{amsfonts}
\usepackage{amssymb}
\usepackage[version=4]{mhchem}
\usepackage{fancyhdr}
\usepackage{setspace}
\usepackage{graphicx}              
\usepackage[margin=31mm,paperheight=237mm,paperwidth=173mm,top=0.95in]{geometry}
\usepackage{ragged2e}
\usepackage{stmaryrd}
\usepackage{CJKutf8}

\justifying

\setcounter{page}{24}
\pagestyle{fancy}
\fancyhf{}
\renewcommand{\headrulewidth}{0pt}
\renewcommand{\footrulewidth}{0pt}
\renewcommand{\baselinestretch}{0.75} 
\fontsize{10pt}{14pt}\selectfont
\setstretch{0.9}

\begin{document}
\setlength{\abovedisplayskip}{3pt}
\setlength{\belowdisplayskip}{3pt}
\setlength{\abovedisplayshortskip}{3pt}
\setlength{\belowdisplayshortskip}{3pt}

Согласно определению натуральных чисел, каждое натуральное число $n$ представимо в виде
\begin{equation}
  n=(\ldots((1+1)+1 \ldots+1)+1 . \tag{1.4}
\end{equation}

\noindent (Рассматривая множества, состоящие из «одинаковых» элементов, можно сказать, что в формуле в правой части стоит множество из $n$ единиц, соединенных последовательно знаком «+».)

Согласно конструкции, определяющей натуральные числа, число $n+1 \in N$ характеризуется тем свойством, что каждое множество, состоящее из $n+1$ элемента, после удаления любого из них превращается в множество, которое состоит из $n$ элементов.

Согласно той же конструкции, любые два множества, состоящие из $n \in N$ элементов, отображаются друг на друга взаимно однозначно.

Если из двух натуральных чисел $m$ и $n$ число $m$ встречается раньше, чем $n$ в ряде натуральных чисел (1.3), т. е. число $m$ стоит левее числа $n$, то число $m$ называют меньшим числа $n$ и пишут $m<n$ или, что то же, число $n$ называют большим числа $m$ и пишут $n>m$.

Например, $n-1<n<n+1, n \neq 1$. Для любых двух различных натуральных чисел $m$ и $n$ имеет место точно одно из соотношений $m<n$ или $m>n$. При этом если $m<n$ и $n<p$, то $m<p, m, n, p \in N$.

Если число $m \in \boldsymbol{N}$ меньше числа $n \in \boldsymbol{N}$, то в каждом множестве, состоящем из $n$ элементов, имеются подмножества, состоящие из $m$ элементов. Это следует из самой конструкции последовательного определения натуральных чисел.

Множество натуральных чисел $\boldsymbol{N}$ обладает следующим замечательным свойством.

\textit{Если множество М таково, что:}

1) $\boldsymbol{M} \subset \boldsymbol{N;}$

2) $1 \in \boldsymbol{M;}$

3) из $n \in \boldsymbol{M}$ следует, что $n+1 \in \boldsymbol{M}$, то
\begin{equation}
  \boldsymbol{M}=\boldsymbol{N} \tag{1.4}
\end{equation}

Действительно, согласно условию 2), имеем $1 \in \boldsymbol{M}$, поэтому, согласно свойству 3$)$, и $2 \in \boldsymbol{M}$; тогда, согласно тому же свойству 3), получаем $3 \in \boldsymbol{M}$. Но любое натуральное число $n \in \boldsymbol{N}$ получается из 1 последовательным прибавлением к ней той
\begin{center}
\rule{0.75in}{0.4pt} \\ \textit{\thepage}
\end{center}

\end{document}